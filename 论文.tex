\documentclass[a4paper]{ctexart}
\usepackage[left=2.50cm, right=2.50cm, top=2.50cm, bottom=2.50cm]{geometry} %页边距

\usepackage{ctex}
\usepackage{amsmath}

\title{结合慕测(mooctest)平台的后台数据对学生编程能力分析报告}

\author{\small 戴俊浩,郭增嘉,刁苏阳}
\renewcommand{\baselinestretch}{1.5}
\date{\small 第一次创建:2020年7月20号\\最后修改时间:2020年7月20号}
\begin{document}
\ttfamily \songti
\zihao {-4}
\maketitle
\section{研究问题:学生编程能力探究}
\subsection{详细介绍}
我们将研究方向设定为学生编程能力的分析,我们通过主成分分析法对平台的数据进行题目难度分析
然后结合分析的题目难度,将学生的编程得分等因素综合考量,使用相关性检测之后将高度重复的
变量除去,最后使用多元线性回归对高维数据进行分析,产生可视化的三维函数,得出一个学生的
编程能力的计算公式,当学生答题产生数据后就能通过相应的算法来得出这个学生的编程能力。
\subsection{研究背景}
\subsection{应用场景与价值}
\section{研究方法}
\subsection{数据源}
使用了慕测平台的数据,分析学生的编程能力,通过OJ提供的json文件可以看到学生的userID以及一系列的提交记录,使用python读取json文件获取所需数据。
\subsection{数据分析}
\subsubsection{主成分分析(PCA)}
在处理题目难度的时候使用了主成分分析法,首先将题目的数据通过对json文件的分析导入到一个矩阵中,这个矩阵是一个3000*7的矩阵,7列是题目的7个维度的数据
,分别是该题的初始提交分,该题的最终得分,该题的平均分,该题的最终得分和提交次数的比值,该题的前一半的提交的斜率,也就是用户在做题时分数的增长
速率还有该题的总耗时以及该题的最终分数和其他题目的最终分数的差值。
\subsubsection{相关性分析(pearson)与显著性检验}
\subsubsection{多元线性回归}
$$X^TXB=X^TY$$
\subsubsection{数据可视化}
\section{代码解析}
\subsection{代码逻辑}
\subsubsection{题目难度分析模块}
\subsubsection{学生编程能力分析模块}
\subsubsection{数据可视化}
\subsection{代码开源地址}
\section{案例分析}
\subsection{分析题目难度}
\subsection{分析学生编程能力}
\section{项目前瞻:学生学习路径推荐}
\section{对老师的意见和建议}
换一个老师见鬼了
\section{附录}
\subsection{数据}
\subsection{图表}
\section{参考文献}

使用$\\LaTeX$进行文本编辑
\end{document}